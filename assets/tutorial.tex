\documentclass{article}

\usepackage[textwidth=16cm]{geometry}
\usepackage[T1]{fontenc}
\usepackage[utf8]{inputenc}
\usepackage{tikz}
\usepackage{xcolor}
\usepackage{upquote}
\usepackage[implicit=false]{hyperref}
\hypersetup{
    colorlinks=true,
    linkcolor=blue,
    filecolor=blue,
    urlcolor=blue,
    pdftitle={Overleaf Example},
    pdfpagemode=FullScreen,
	linktoc=None
    }

\makeatletter
\g@addto@macro\thesection.
\makeatother


\usetikzlibrary{backgrounds}
\usetikzlibrary{calc}

\date{}
\newcommand*\keystroke[1]{%
  \begin{tikzpicture}[baseline=(key.base), very thin, line cap=round, black, rounded corners=0pt]%
    \node [draw, fill=white, fill opacity=1, rectangle, rounded corners=2pt, inner sep=1pt, minimum width=1.2em, font=\scriptsize\sffamily] (key) {#1\strut};

    \begin{scope}[on background layer]
      \draw [rounded corners=1pt, fill=white] ($ (key.north west) + (-2pt, 2pt) $) rectangle ($ (key.south east) + (2pt, -2pt) $);

      \fill [gray!60] ($ (key.south west) + (2pt, 0.1pt) $) -- ($ (key.south west) + (-1pt, -2pt) $)
                  -- ($ (key.south east) + (1pt, -2pt) $)  -- ($ (key.south east) + (-2pt, 0.1pt) $) -- cycle;

      \fill [gray!60] ($ (key.south east) + (-0.1pt, 2pt) $) -- ($ (key.south east) + (2pt, -1pt) $)
                  -- ($ (key.north east) + (2pt, 1pt) $)    -- ($ (key.north east) + (-0.1pt, -2pt) $) -- cycle;
    \end{scope}

    \draw ($ (key.north west) + (0.1pt, -2pt) $) -- ($ (key.north west) + (-2pt, 1pt) $);
    \draw ($ (key.north west) + (2pt, -0.1pt) $) -- ($ (key.north west) + (-1pt, 2pt) $);

    \draw ($ (key.north east) + (-0.1pt, -2pt) $) -- ($ (key.north east) + (2pt, 1pt) $);
    \draw ($ (key.north east) + (-2pt, -0.1pt) $) -- ($ (key.north east) + (1pt, 2pt) $);

    \draw ($ (key.south west) + (0.1pt, 2pt) $) -- ($ (key.south west) + (-2pt, -1pt) $);
    \draw ($ (key.south west) + (2pt, 0.1pt) $) -- ($ (key.south west) + (-1pt, -2pt) $);

    \draw ($ (key.south east) + (-0.1pt, 2pt) $) -- ($ (key.south east) + (2pt, -1pt) $);
    \draw ($ (key.south east) + (-2pt, 0.1pt) $) -- ($ (key.south east) + (1pt, -2pt) $);
  \end{tikzpicture}%
}




\title{vi\_Sioyek Manual}
\begin{document}
\maketitle

\section{Navigation}
\begin{itemize}
	\item Use \keystroke{h} \keystroke{j} \keystroke{k} \keystroke{l} keys for navigation.
	\item Press \keystroke{d} / \keystroke{u} to scroll half Down/Up.
	\item Press \keystroke{D} / \keystroke{U} to go to previous/next page.
	\item Press \keystroke{J} / \keystroke{K} to zoom in/out.
	\item Enter \keystroke{gg} to go to the first page.
	\item Press \keystroke{G} to go to the last page.
	\item Enter [Number]+\keystroke{g} to jump to a page.
	\item Press \keystroke{s} to fit the document to screen (including white margins).
	\item Press \keystroke{S} to fit the document to screen (ignoring white page margins).
	\item Press \keystroke{Tab} to open table of contents (if it exists).
	\item Press \keystroke{F11} to enter fullscreen mode.
	\item Press \keystroke{i} to toggle dark mode.

\end{itemize}
\section{Manage Files}
\begin{itemize}
	\item Press \keystroke{o} to open a new document
	\item Press \keystroke{Shift}+\keystroke{o} to open a previously opened document.
\end{itemize}

\section{Search}
\begin{itemize}
	\item Press \keystroke{/} to search.
	\item Press \keystroke{n} / \keystroke{Shift}+\keystroke{n} to go to the next/previous match.
	\item In order to quickly jump in search results, you can use \verb+[NUM]+\keystroke{n} where \verb+[NUM]+ is a number. For example in order to jump to the 10th next occurance of current search term, enter \keystroke{10n}. In fact most commands in sioyek can be prefixed with a number \verb+N+ which is equivalent to repeating them \verb+N+ times.

\bigskip
With some text selected, press:
	\item	\keystroke{w}+\keystroke{s}+\keystroke{g} to search the text in Google.
	\item	\keystroke{w}+\keystroke{s}+\keystroke{s} to search the text in Google Scholar.
	\item	\keystroke{w}+\keystroke{s}+\keystroke{l} to search it in Library Genesis.
\end{itemize}


\section{Command Palette}
\begin{itemize}
	\item Press \keystroke{:} to open a searchable list of all sioyek commands
	\item If the command is bound to a key, the keybinding is shown on the right (multiple keybinds are separated by | symbol)
\end{itemize}
\section{Marks}
\begin{itemize}
	\item Press \keystroke{m} to set a mark in current location. After you press m, sioyek waits for you to press another symbol which is the name of the mark. For example to create a mark named u, enter \keystroke{mu}.
	\item In order to go to a mark, press \keystroke{\`} (backtick) followed by the name of the mark. For example in order to go to the mark in previous example, we enter \keystroke{\`\ u}. Note that the backtick character \keystroke{\`} is different from the single quote character \keystroke{'}.
	\item Note that marks are persistent (they are saved even after sioyek is closed). Lower case marks are local to each file, while upper case marks are global.
	\item You can right click on any line to create a visual mark which highlights below the line. This highlight has multiple functions.
	 First, it can reduce disorientation when scrolling by first right clicking the line we are reading and then scrolling. Second, it can also act as a mark, so we can jump back to the location where we last put the highlight by pressing \keystroke{\`} and then right clicking. You can try it out here:\\
	  {\color{red} Right click on me!}
	\item You can move this visual mark to the next/previous line by pressing \keystroke{j} and \keystroke{k}. You can also press \keystroke{F7} to enable visual scroll mode. In this mode, scrolling the mouse wheel moves this visual mark.

\end{itemize}

\section{Bookmarks}
\begin{itemize}
	\item Press \keystroke{b} to create a bookmark in the current location.
	\item Press \keystroke{gb} to search the bookmarks in the current file and \keystroke{gB} (that is g followed by capital B) to search all the bookmarks.
	\item In order to delete a bookmark, first goto the bookmark and then enter \keystroke{db} (delete the closest bookmark).
	\item If you select a piece of text and then press \keystroke{b} then the selected text will automatically be used as the bookmark text. You can try it on the following text:\\
	{\color{red} Bookmark me!}

\end{itemize}

\section{Highlights}
\begin{itemize}
	\item Select a piece of text and then press \keystroke{H} followed by a lower case letter to highlight the selected text (the letter can be though of as the "type" of the highlight). For example you can press \keystroke{hh} to create a highlight of type "h". Different types of highlight are colored differently, and the color is configurable in the \verb+prefs.config+ file.
	\item Press \keystroke{gh} to search the highlights in the current file and \keystroke{gH} (that is g followed by capital H) to search all the highlights.
	\item In order to delete a highlight, you can click on a highlight and then press \keystroke{dh}.

\end{itemize}

\section{Smart Jump}


\begin{itemize}
	\item You can middle click on figure names (for example on `1' in `Figure 1') to jump to the location where that figure is located. This works even if the document is not linked. You can also middle click on in-line references (for example on `2' in `\cite{branner1989mandelbrot,mandelbrot2004fractals}') to go to the corresponding entry in the "References" section.
	\item You can also right click on figure names or in-line references to open a quick overview of their location instead of jumping to them.
	\item Try it out by middle / right clicking on the figure name / in-line reference in the following text :\\
	 You can see a beautiful colorization of mandlebrot set in Figure \ref{fig:mandle}. For more information about the mandlebrot set, see \cite{branner1989mandelbrot}.
	\item You can also middle click on the full-citation of a paper to search it in Google Scholar, or you can  \keystroke{shift}+middle click on it to search it in Library Genesis. You can use this feature, for example, while in the "References" section of a document.
\end{itemize}


\section{Portals}
\begin{itemize}
	\item Press \keystroke{F12} to open the helper window. This window automatically displays the closest portal to the current location.
	\item In order to create a portal, press \keystroke{p} to enter the current location as the source of the portal. Now you navigate to the portal destination and press \keystroke{p} again to mark that location as the destination of the portal.
	 (note that source and destination don't have to be in the same document, for example you can create a portal from a reference to a paper to the actual paper).
	\item You can also create a portal by pressing \keystroke{p} and then clicking (or middle clicking using smart jumps) on a link. This automatically creates a link between the current location and the destination of the link.
	\item In order to delete a portal, press \keystroke{dp}. This deletes the closest portal to the current location (the portal that is being displayed in the helper window).
	\item Sometimes you need to slightly edit a portal (especially when creating a portal using links, because sometimes for example a link to a figure doesn't center the figure properly in the screen). In order to do this, press \keystroke{shift}+\keystroke{p}(\keystroke{P}) while the portal is active. This takes you to the portal's destination. Now you can adjust the destination and once you are done press the back button (by default it is \keystroke{backspace}) to go back to where you were.
	\item You can also use the mouse wheel to adjust the portal in helper window. You can also pan it using mouse drag or zoom using mouse wheel while holding control.
\end{itemize}


\section{Configuration}
\begin{itemize}
\item There are four configuration files: two of them are system-wide and are not meant to be edited by the user (\verb+keys.config+ and \verb+prefs.config+), while the other two are user-editable (\verb+keys_user.config+ and \verb+prefs_user.config+).
\item The former (i.e., the system-wide configuration files) are always loaded by sioyek before the latter. This implies that configuration options set in the user-editable configuration files always overwrite the ones in the system-wide files.
\item In order to edit any configuration, copy the configuration line you want to edit to the user-editable config file and change it there. \\ For example, suppose we want to change the key that opens the table of contents. We search for `table of contents' in \verb+keys.config+ and find the following:
\begin{verbatim}
# Open table of contents.
goto_toc t
\end{verbatim}
We copy this line to \verb+keys_user.config+ and change the key. For example, suppose we want to use capital \keystroke{T} to open the table of contents. We can put the following line in \verb+keys_user.config+:
\begin{verbatim}
goto_toc T
\end{verbatim}
\item In portable builds, the config files are located in the same place where \verb+sioyek+ executable file is located. In non-portable builds, their location is OS-dependent. You can also open them by pressing \keystroke{:} to open the command window and then entering one of the following:
\begin{itemize}
	\item \verb+prefs+
	\item \verb+prefs_user+
	\item \verb+keys+
	\item \verb+keys_user+
\end{itemize}
\end{itemize}

\section{Synctex}

\begin{itemize}
	\item Press \keystroke{F4} to enable "synctex mode". While in synctex mode, right clicking on a piece of texts opens the corresponding latex file location.
	\item You can configure the synctex inverse search command using \verb+inverse_search_command+ config in \verb+prefs.config+ . Here is an example for VsCode:

\begin{verbatim}
inverse_search_command 		"C:\path\to\vscode\Code.exe" -r -g %1:%2
\end{verbatim}
\%1 stands for the name of the file and \%2 stands for the line number in the file.
\item Here is the forward search configuration for Latex Workshop for VsCode:
\begin{verbatim}
"latex-workshop.view.pdf.external.synctex.command": "C:\\path\\to\\sioyek.exe",
"latex-workshop.view.pdf.external.synctex.args": [
	"--inverse-search",
	"\"C:\\path\\to\\vscode\\Code.exe\" -r -g %1:%2",
	"--reuse-window",
	"--forward-search-file",
	"%TEX%",
	"--forward-search-line",
	"%LINE%",
	"%PDF%"
]
\end{verbatim}
And here is example configuration for VimTeX:

\begin{verbatim}
let g:vimtex_view_method = 'sioyek'
\end{verbatim}
\end{itemize}

\section{Extensions}
You can extend sioyek by writing your own extensions. See \href{https://sioyek-documentation.readthedocs.io/en/latest/scripting.html}{the documentation} on how to do it.
We also have some extensions \href{https://github.com/ahrm/sioyek-python-extensions}{available here}. For example we have an extension that downloads the paper from google scholar just by control+clicking on its name.

\bibliographystyle{unsrt}
\bibliography{bibs}

\end{document}
